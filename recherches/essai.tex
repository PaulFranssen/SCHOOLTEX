

\documentclass[a4paper, 12pt]{article}
\usepackage{lmodern}
\usepackage[french]{babel}
\usepackage[T1]{fontenc}
\usepackage[utf8]{inputenc}

\title{\Large \textbf{LARGE ET GRAS RAPPORT}}
\author{Paul FRANSSEN}
\date{\today}



\begin{document}

    \begin{titlepage}
        \maketitle
    \end{titlepage}
    
    \setcounter{page}{2}

    \tableofcontents

    \newpage

    \section{Introduction}
    \subsection{Pourquoi ce projet ?}
    %\paragraph{}

    Voir note bas de page\footnote{projet d'année} pour évaluer la validité de l'hypothèse 39, pourquoi chercher?
\\ %passage à la ligne
    \'Eviter accentuation majuscule % espace pour un paragraphe

    Vecteur : \overrightarrow{ABC} 

    \vspace{3\baselineskip} 
    espacement de 3 lignes

    arbrisseau

    \subsection{Résolution des problèmes}
    Détail de la chambre :
    \begin{itemize}
        \item un lit ;
        % Saut de ligne optionnel (aérer)
        \item une armoire ; 
        \item et un bureau. \\
        % Saut de ligne (\\) licite pour
        aérer le texte
        \end{itemize}

        \vspace{2\baselineskip} 
        Pour écrire sous \LaTeX{}, il
        faut :
        \begin{enumerate}
        \item Apprendre les bases.
        \item Pratiquer les bases.
        \item \^Etre curieux !
        \end{enumerate}

    %aller à la ligne (laisser une ligne)
    ruisseau
    \\passage ligne sans retrait paragraphe
    
    \og --- oui \fg{}

    \newpage
    Nouvelle Page
    
\end{document}
