\documentclass[utf8, a4paper, 12 pt]{article}
\usepackage{graphicx, float}
\usepackage{fourier}
\usepackage[french]{babel}
\usepackage[T1]{fontenc}
\usepackage[utf8]{inputenc}
\usepackage{enumitem, pifont}
\usepackage{amsmath, amsfonts,amssymb}   
\usepackage{tikz}
\usepackage{array}
\usepackage{lscape}                                                                                                                                                                                                                                                                                                                                                                                                                                                                                                                                                                                                                                                                                                                                                                                                                                                                                                                                                                                                                                                                                                                                                                                                                                                                                                                                                                                                                                                                                                                                                                                                                                                                                                                                                                                                                                                                                                                                                                                                                                                                                                                                                                                                                                                                                                                                                                                                  

\title{\LARGE \textbf{MESURES ET CALCULS DE SURFACES}}
\author{MATH\'EMATHIQUES 6P}
\date{\today}
\setlist[itemize, 1]{label ={\ding{248}}, itemsep = \baselineskip}



\begin{document}

    \large
    \everymath{\displaystyle} 
    \maketitle
    \setcounter{page}{2}

    \section*{Mesure et calcule les surfaces suivantes :}

        Réponses au dixième de $cm^2$ près

        \vspace{1\baselineskip}

        \subsection*{Surface 1}

            \vspace{1\baselineskip}
            
            \begin{tikzpicture}
                \draw circle(3.1);
            \end{tikzpicture}
            \vspace{1\baselineskip}

        \subsection*{Surface 2}
            \vspace{1\baselineskip}
            \begin{tikzpicture}
                \draw (0,0) rectangle (5,7);
            \end{tikzpicture}
            \vspace{1\baselineskip}

        \subsection*{Surface 3}
            \vspace{1\baselineskip}
            \begin{tikzpicture}
                \draw (0,0) -- (6,0) --++ (120:5) -- cycle; 
            \end{tikzpicture}
            \vspace{1\baselineskip}

        \subsection*{Surface 4}
            \vspace{1\baselineskip}
            \begin{tikzpicture}
                \draw (0,0) -- (5,0) --++ (90:4) --++(180:2) -- cycle; 
            \end{tikzpicture}
            \vspace{1\baselineskip}

        \subsection*{Surface 5}
            \vspace{1\baselineskip}
            \begin{tikzpicture}
                \draw (0,0) -- (8,0) --++ (90:2)  --++ (140:3) --++(180:3) -- cycle; 
            \end{tikzpicture}

        \subsection*{Surface 6}
            \vspace{1\baselineskip}
            \begin{tikzpicture}
                \draw (0,0) -- (2,2) arc(180:0:2) -- (8,0) -- cycle; 
            \end{tikzpicture}

\end{document}
